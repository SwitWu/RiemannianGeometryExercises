\documentclass[a4paper]{article}
\title{\bfseries 黎曼几何练习题}
\author{wsy}
\date{\today}
\usepackage{ctex}
\usepackage{mathtools}
\usepackage{amssymb}
\usepackage{amsthm}
\usepackage{mathrsfs}
\usepackage[shortlabels]{enumitem}
\usepackage{physics}
\usepackage{yhmath}

\setCJKmainfont{FandolSong}
\newcommand{\diff}{\mathop{}\!\mathrm{d}}
\renewcommand{\div}{\operatorname{div}}
\setlist{nosep, left=\parindent}
\DeclarePairedDelimiterX{\innerp}[2]{\langle}{\rangle}{#1,#2}
% \DeclareMathOperator{\tr}{tr}
\DeclareMathOperator{\Hess}{Hess}
\DeclareMathOperator{\Ric}{Ric}
\DeclareMathOperator{\D}{D}
\DeclareMathOperator{\Span}{span}
\DeclareMathOperator*{\esssup}{ess\,sup}

\makeatletter
\renewenvironment{proof}[1][\proofname]{\par
	\pushQED{\qed}%
	\normalfont\topsep1\p@\@plus6\p@\relax
	\trivlist
	\item\relax
	{\hspace*{\parindent}\textbf{#1}\@addpunct{:}}\hspace\labelsep\ignorespaces
}{%
	\popQED\endtrivlist\@endpefalse
}
\makeatother

\newcounter{exercise}
\NewDocumentEnvironment{exercise}{o +b}
  {
    \IfNoValueTF{#1}
      {\stepcounter{exercise}}
      {\setcounter{exercise}{#1}}
    \par\textbf{\theexercise. }#2
  }
  {}

\begin{document}

\maketitle

\section{练习题一}



\begin{exercise}
  设 $(M,g)$ 是 $n$ 维可定向黎曼流形, $X\in\mathscr{X}(M)$, $(U;x^i)$
  是定向相符的局部坐标系, 令
  \[\omega=\sum_{i=1}^n (-1)^{i+1}\sqrt{G}X^i\diff x^1\wedge\cdots\wedge\widehat{\diff x^i}\wedge\cdots\wedge\diff x^n,\]
  其中 $G=\det(g_{ij})$, $X=X^i\frac{\partial}{\partial x^i}$. 证明
  \begin{enumerate}[(1)]
    \item $\omega$ 是 $M$ 上整体定义的 $n-1$ 次外微分式.
    \item $i_X\diff V_M=\omega$, 其中 $i_X$ 为内乘, 定义为对任 $n$ 次外微分式 $\varphi$,
      $\forall X_1,\cdots,X_{n-1}\in\mathscr{X}(M)$ 有
      \[i_X\varphi(X_1,\cdots,X_{n-1})=\varphi(X,X_1,\cdots,X_{n-1}).\]
  \end{enumerate}
\end{exercise}

\begin{proof}
  (1) 设 $(V;y^i)$ 是另一个定向相符的局部坐标系, 则
  \[X=X^i\frac{\partial}{\partial x^i}=Y^i\frac{\partial}{\partial y^i}
    =Y^i\frac{\partial x^j}{\partial y^i}\frac{\partial}{\partial x^j},\]
    于是 $X^j=Y^i\frac{\partial x^j}{\partial y^i}$. 记
    \[\widetilde{G} = \det\biggl(g\biggl(\pdv{y^i},\pdv{y^j}\biggr)\biggr),\]
    则
    \[G = \biggl(\pdv{(y^1,\cdots,y^n)}{(x^1,\cdots,x^n)}\biggr)^2\widetilde{G}.\]
    将上述结果代入 $\omega$ 的表达式得
    \begin{align*}
      & (-1)^{i+1} \sqrt{G}X^i\diff x^1\wedge\cdots\wedge\widehat{\diff x^i}\wedge\cdots\wedge\diff x^n \\
      ={} & (-1)^{i+1}\sqrt{\biggl(\frac{\partial(y^1,\cdots,y^n)}
        {\partial(x^1,\cdots,x^n)}\biggr)^2\widetilde{G}} \cdot Y^k\frac{\partial x^i}{\partial y^k}
        \biggl(\frac{\partial x^1}{\partial y^{j_1}}\diff y^{j_1}\biggr)
        \wedge\cdots\wedge\widehat{\diff x^i}\wedge\cdots\wedge
        \biggl(\frac{\partial x^n}{\partial y^{j_n}}\diff y^{j_n}\biggr) \\
      ={} & (-1)^{i+1}\frac{\partial(y^1,\cdots,y^n)}{\partial(x^1,\cdots,x^n)}
        \sqrt{\widetilde{G}}Y^k\frac{\partial x^i}{\partial y^k}
        \frac{\partial x^1}{\partial y^{j_1}}\cdots\widehat{\frac{\partial x^i}{\partial y^{j_i}}}
        \cdots\frac{\partial x^n}{\partial y^{j_n}}
        \diff y^{j_1}\wedge\cdots\wedge\widehat{\diff y^{j_i}}\wedge\cdots\wedge\diff y^{j_n}.
    \end{align*}
    对于固定的 $i$,
    \begin{align*}
      & Y^k\frac{\partial x^i}{\partial y^k}
        \frac{\partial x^1}{\partial y^{j_1}}\cdots\widehat{\frac{\partial x^i}{\partial y^{j_i}}}
        \cdots\frac{\partial x^n}{\partial y^{j_n}}
        \diff y^{j_1}\wedge\cdots\wedge\widehat{\diff y^{j_i}}\wedge\cdots\wedge\diff y^{j_n}. \\
      ={} & Y^i \frac{\partial(x^1,\cdots,x^n)}{\partial(y^1,\cdots,y^n)}
        \diff y^1\wedge\cdots\wedge\widehat{\diff y^i}\wedge\cdots\wedge\diff y^n.
    \end{align*}
    因此
    \begin{align*}
      \omega
      & = \sum_{i=1}^n (-1)^{i+1} \frac{\partial(y^1,\cdots,y^n)}{\partial(x^1,\cdots,x^n)}
        \sqrt{\widetilde{G}} Y^i \frac{\partial(x^1,\cdots,x^n)}{\partial(y^1,\cdots,y^n)}
        \diff y^1\wedge\cdots\wedge\widehat{\diff y^i}\wedge\cdots\wedge\diff y^n \\
      & = \sum_{i=1}^n (-1)^{i+1} \sqrt{\widetilde{G}}Y^i \diff y^1\wedge\cdots\wedge\widehat{\diff y^i}\wedge\cdots\wedge\diff y^n.
    \end{align*}
    即 $\omega$ 与坐标系的选取无关, 是大范围定义的几何量.

    (2) 由内乘运算的性质, 对于 $\forall f\in C^{\infty}(M)$, $\varphi\in A^k(M)$, 有
    \[i_X (f\varphi) = f i_X\varphi.\]
    \[i_X (\varphi\wedge\psi) = i_X\varphi\wedge\psi+(-1)^k \varphi\wedge i_X\psi.\]
    故
    \begin{align*}
      i_X \diff V_M
      & = i_X \Bigl(\sqrt{G}\diff x^1\wedge\cdots\wedge\diff x^n\Bigr) \\
      & = \sum_{i=1}^n \sqrt{G} (-1)^{i+1} \diff x^1\wedge\cdots\wedge\diff x^i(X)\wedge\cdots\diff x^n \\
      & = \sum_{i=1}^n (-1)^{i+1} \sqrt{G}X^i\diff x^1\wedge\cdots\wedge\widehat{\diff x^i}\wedge\cdots\diff x^n \\
      & = \omega.\qedhere
    \end{align*}
\end{proof}



\begin{exercise}
设 $M$ 是嵌入在 $\mathbb{R}^{n+1}$ 中的超曲面, $(x^A)$ 是 $\mathbb{R}^{n+1}$
中的直角坐标系. 对于任意的点 $p\in M$, 存在 $p$ 在 $\mathbb{R}^{n+1}$ 中的开邻域 $U$, 使得
$M\cap U$ 有参数表示
\[x^A=f^A(u^1,\cdots,u^n),\quad 1\leq A\leq n+1,\quad (u^1,\cdots,u^n)\in D\subset\mathbb{R}^n,\]
其中 $D$ 是 $\mathbb{R}^n$ 中的开邻域.

(1) 证明: $M$ 上的单位法向量场 $\xi$ 的分量是 $\xi^A=W^A/W$, 其中
\[W^A=(-1)^{A+1}\frac{\partial (f^1,\cdots,\widehat{f^A},\cdots,f^{n+1})}{\partial(u^1,\cdots,u^n)},
  \quad W=\biggl(\sum_A (W^A)^2\biggr)^{1/2};\]

(2) 求 $\mathbb{R}^{n+1}$ 在 $M$ 上的诱导黎曼度量 $g=\sum g_{ij}\diff u^i\diff u^j$, 并证明
$G=\det(g_{ij})=W^2$;

(3) 证明: $M$ 的体积元素是
\[\diff V_M = i(\xi)(\diff x^1\wedge\cdots\wedge\diff x^{n+1})|_M.\]
\end{exercise}

\begin{proof}
  (1) $M$ 的自然标架场为
  \[e_i=\frac{\partial}{\partial u^i}=\frac{\partial f^A}{\partial u^i}\frac{\partial}{\partial x^A}
    =\biggl(\frac{\partial f^1}{\partial u^i},\cdots,\frac{\partial f^{n+1}}{\partial u^i}\biggr),
    \quad (i=1,\cdots,n).\]
  而
  \begin{align*}
    0
    ={} & \begin{vmatrix}
      \frac{\partial f^1}{\partial u^1} & \cdots & \frac{\partial f^1}{\partial u^n} & \frac{\partial f^1}{\partial u^i} \\
      \vdots & \ddots & \vdots & \vdots \\
      \frac{\partial f^{n+1}}{\partial u^1} & \cdots & \frac{\partial f^{n+1}}{\partial u^n} & \frac{\partial f^{n+1}}{\partial u^i}
    \end{vmatrix} \\
    ={} & \frac{\partial f^1}{\partial u^i}(-1)^{n+2}
      \begin{vmatrix}
        \frac{\partial f^2}{\partial u^1} & \cdots & \frac{\partial f^2}{\partial u^n} \\
        \vdots & \ddots & \vdots \\
        \frac{\partial f^{n+1}}{\partial u^1} & \cdots & \frac{\partial f^{n+1}}{\partial u^n}
      \end{vmatrix}
      +
      \frac{\partial f^2}{\partial u^i}(-1)^{n+3}
      \begin{vmatrix}
        \frac{\partial f^1}{\partial u^1} & \cdots & \frac{\partial f^1}{\partial u^n} \\
        \frac{\partial f^3}{\partial u^1} & \cdots & \frac{\partial f^3}{\partial u^n} \\
        \vdots & \ddots & \vdots \\
        \frac{\partial f^{n+1}}{\partial u^1} & \cdots & \frac{\partial f^{n+1}}{\partial u^n}
      \end{vmatrix} \\
    & + \cdots +
      \frac{\partial f^{n+1}}{\partial u^i}(-1)^{2n+2}
      \begin{vmatrix}
        \frac{\partial f^1}{\partial u^1} & \cdots & \frac{\partial f^1}{\partial u^n} \\
        \vdots & \ddots & \vdots \\
        \frac{\partial f^n}{\partial u^1} & \cdots & \frac{\partial f^n}{\partial u^n}
      \end{vmatrix} \\
    ={} & (-1)^n \biggl(\frac{\partial f^1}{\partial u^i} W^1 + 
      \frac{\partial f^2}{\partial u^i} W^2 + \cdots +
      \frac{\partial f^{n+1}}{\partial u^i} W^{n+1}\biggr) \\
    ={} & (-1)^n \langle e_i,\xi\rangle W.
  \end{align*}
  因此 $\xi$ 是法向量且 $W=\Bigl(\sum_A (W^A)^2\Bigr)^{1/2}$.

  (2) 由于
  \[g_{ij}=\innerp{e_i}{e_j} = \innerp*{\frac{\partial f^A}{\partial u^i}\frac{\partial}{\partial x^A}}
                                      {\frac{\partial f^B}{\partial u^j}\frac{\partial}{\partial x^B}}
    = \frac{\partial f^A}{\partial u^i}\frac{\partial f^A}{\partial u^j}.\]
  故
  \begin{align*}
    \det(g_{ij})
    & = \begin{vmatrix}
      \frac{\partial f^A}{\partial u^1}\frac{\partial f^A}{\partial u^1} &
      \cdots &
      \frac{\partial f^A}{\partial u^1}\frac{\partial f^A}{\partial u^n} \\
      \vdots & \ddots & \vdots \\
      \frac{\partial f^A}{\partial u^n}\frac{\partial f^A}{\partial u^1} &
      \cdots &
      \frac{\partial f^A}{\partial u^n}\frac{\partial f^A}{\partial u^n}
      \end{vmatrix} \\
    & = \begin{vmatrix}
      \begin{pmatrix}
        \frac{\partial f^A}{\partial u^1} \\
        \vdots \\
        \frac{\partial f^A}{\partial u^n}
      \end{pmatrix}
      \begin{pmatrix}
        \frac{\partial f^A}{\partial u^1} & \cdots & \frac{\partial f^A}{\partial u^n}
      \end{pmatrix}
    \end{vmatrix} \\
    &  = \det\begin{Bmatrix}
      \begin{pmatrix}
        \frac{\partial f^1}{\partial u^1} & \cdots & \frac{\partial f^{n+1}}{\partial u^1} \\
        \vdots & \ddots & \vdots \\
        \frac{\partial f^1}{\partial u^n} & \cdots & \frac{\partial f^{n+1}}{\partial u^n}
      \end{pmatrix}
      \begin{pmatrix}
        \frac{\partial f^1}{\partial u^1} & \cdots & \frac{\partial f^1}{\partial u^n} \\
        \vdots & \ddots & \vdots \\
        \frac{\partial f^{n+1}}{\partial u^1} & \cdots & \frac{\partial f^{n+1}}{\partial u^n}
      \end{pmatrix}
    \end{Bmatrix}.
  \end{align*}
  由于 $M$ 是嵌入超曲面, 故 $\frac{\partial (f^1,\cdots f^{n+1})}{\partial (u^1,\cdots,u^n)}$
  是秩为 $n$ 的矩阵, 其必有某一行可由其余行线性表示, 因此可取局部坐标系 $(u^1,\cdots,u^n)$ 使得
  $\frac{\partial f^{n+1}}{\partial u^i}=0$, $i=1,2,\cdots,n$. 从而
  \[\det(g_{ij}) = \left\lvert\frac{\partial (f^1,\cdots,f^n)}{\partial (u^1,\cdots,u^n)}\right\rvert^2.,\]
  此时
  \[W^A = \begin{cases}
    \frac{\partial (f^1,\cdots,f^n)}{\partial (u^1,\cdots,u^n)}, & A=n+1, \\
    0, & A\neq n+1.
  \end{cases}\]
  因此 $\det(g_{ij})=|W|^2=G$.

  (3)
  \begin{align*}
    & i_{\xi}(\diff x^1\wedge\cdots\wedge\diff x^{n+1}) \\
    ={} & (-1)^{A+1}\xi^A\diff x^1\wedge\cdots\widehat{\diff x^A}\wedge\cdots\wedge\diff x^{n+1}|_M \\
    ={} & (-1)^{A+1}\frac{W^A}{A} \frac{\partial x^1}{\partial u^{i_1}}
      \cdots\widehat{\frac{\partial x^A}{\partial u^{i_A}}}
      \frac{\partial x^{n+1}}{\partial u^{i_{n+1}}}
      \diff u^{i_1}\wedge\cdots\wedge\widehat{\diff u^{i_A}}\wedge\cdots\wedge\diff u^{i_{n+1}} \\
    ={} & (-1)^{A+1}\frac{W^A}{A} \frac{\partial x^1}{\partial u^{i_1}}
      \cdots\widehat{\frac{\partial x^A}{\partial u^{i_A}}}
      \frac{\partial x^{n+1}}{\partial u^{i_{n+1}}}
      \delta^{1\cdots n}_{i_1\cdots\widehat{i_A}\cdots i_{n+1}}
      \diff u^1\wedge\cdots\wedge\diff u^n \\
    ={} & (-1)^{A+1}\frac{W^A}{W}
      \begin{vmatrix}
        \frac{\partial x^1}{\partial u^1} & \cdots & \frac{\partial x^1}{\partial u^n} \\
        \vdots & \ddots & \vdots \\
        \frac{\partial x^{n+1}}{\partial u^1} & \cdots & \frac{\partial x^{n+1}}{\partial u^n}
      \end{vmatrix}
      \diff u^1\wedge\cdots\wedge\diff u^n\;
      (\text{行列式中无\ }\frac{\partial x^A}{\partial u^{\cdot}}\text{\ 这一行}) \\
    ={} & W\diff u^1\wedge\cdots\wedge\diff u^n = \diff V_M.\qedhere
  \end{align*}
\end{proof}



\begin{exercise}[4]
  设 $M$ 是 $m$ 维光滑流形, $g$ 和 $\tilde{g}$ 是 $M$ 上的两个黎曼度量.
  如果存在光滑的正函数 $\lambda\in C^{\infty}(M)$, 使得 $\tilde{g}=\lambda^2 g$,
  则 $g$ 和 $\tilde{g}$ 互称为共形的黎曼度量, 简称为共形度量.

  (1) 假定 $(U;x^i)$ 是 $M$ 的容许局部坐标系, 证明: 共形的黎曼度量 $g$ 和 $\tilde{g}$
  的 Christoffel 记号 $\Gamma_{ij}^k$ 和 $\tilde{\Gamma}_{ij}^k$ 满足如下的关系式:
  \[\tilde{\Gamma}_{ij}^k = \Gamma_{ij}^k
    +\delta_i^k \frac{\partial}{\partial x^j}(\ln\lambda)
    +\delta_j^k \frac{\partial}{\partial x^i}(\ln\lambda)
    -g_{ij}g^{kl}\frac{\partial}{\partial x^l}(\ln\lambda).\]
  特别地, 如果 $\lambda=e^{\rho}$, $\rho\in C^{\infty}(M)$, 则上式成为
  \[\tilde{\Gamma}_{ij}^k = \Gamma_{ij}^k
    + \delta_i^k \frac{\partial\rho}{\partial x^j}
    + \delta_j^k \frac{\partial\rho}{\partial x^i}
    -g_{ij}g^{kl}\frac{\partial\rho}{\partial x^l}.\]

  (2) 设 $\Delta_g$ 和 $\Delta_{\tilde{g}}$ 分别是黎曼度量 $g$ 和 $\tilde{g}=\lambda^2g$
  的 Beltrami-Laplace 算子, 利用 (1) 的结论证明:
  \[\Delta_{\tilde{g}} f = \lambda^{-2}
    \bigl(\Delta_g(f) + (m-2)g(\nabla(\ln\lambda),\nabla f)\bigr),\quad\forall f\in C^{\infty}(M).\]
\end{exercise}

\begin{proof}
  (1) 由定义
  \begin{align*}
    \tilde{\Gamma}_{ij}^k
    & = \frac{1}{2}\tilde{g}^{kl}\biggl(\frac{\partial\tilde{g}_{lj}}{\partial x^i}
      + \frac{\partial\tilde{g}_{li}}{\partial x^j}
      - \frac{\partial\tilde{g}_{ij}}{\partial x^l}\biggr) \\
    & = \frac{1}{2}\frac{1}{\lambda^2} g^{kl}
      \biggl(\frac{\partial (\lambda^2 g_{lj})}{\partial x^i}
      + \frac{\partial (\lambda^2 g_{li})}{\partial x^j}
      - \frac{\partial (\lambda^2 g_{ij})}{\partial x^l}\biggr) \\
    & = \frac{1}{2}\frac{1}{\lambda^2} g^{kl}
      \biggl(2\lambda\frac{\partial\lambda}{\partial x^i} g_{lj}
      + 2\lambda\frac{\partial\lambda}{\partial x^j} g_{li}
      - 2\lambda\frac{\partial\lambda}{\partial x^l} g_{ij}\biggr)
      + \Gamma_{ij}^k \\
    & = \Gamma_{ij}^k
      + \delta_i^k \frac{\partial}{\partial x^j}(\ln\lambda)
      + \delta_j^k \frac{\partial}{\partial x^i}(\ln\lambda)
      - g_{ij}g^{kl}\frac{\partial}{\partial x^l}(\ln\lambda).
  \end{align*}

  (2) 由 Laplace 算子定义,
  \[\Delta f = \tr\Hess(f) = g^{ij}(\D\diff f)
    \biggl(\frac{\partial}{\partial x^i},\frac{\partial}{\partial x^j}\biggr).\]
  而
  \[\D\diff f = \D(f_i\diff x^i)
    = \biggl(\frac{\partial f^i}{\partial x^j}-f_k\Gamma_{ij}^k\biggr)\diff x^j\otimes\diff x^i.\]
  故
  \begin{align*}
    \Delta f
    & = g^{ij}\biggl(\frac{\partial f^i}{\partial x^j}-f_k\Gamma_{ij}^k\biggr) \\
    & = \frac{\partial}{\partial x^j}\biggl(g^{ij}\frac{\partial f}{\partial x^i}\biggr)
      - f_i\frac{\partial g^{ij}}{\partial x^j}
      - f_k\Gamma_{ij}^k g^{ij} \\
    & = \frac{\partial}{\partial x^j}\biggl(g^{ij}\frac{\partial f}{\partial x^i}\biggr)
      + \frac{1}{2}g^{kl}\frac{\partial g_{kl}}{\partial x^j}
      g^{ij}\frac{\partial f}{\partial x^i} \\
    & = \frac{1}{\sqrt{G}}\frac{\partial}{\partial x^j}
      \biggl(\sqrt{G} g^{ij}\frac{\partial f}{\partial x^i}\biggr).
  \end{align*}
  故对于 $\tilde{g}=\lambda^2 g$, $\widetilde{G}=\lambda^{2n} G$, 有
  \begin{align*}
    \Delta_{\tilde{g}} f
    & = \frac{1}{\sqrt{\lambda^{2n} G}} \frac{\partial}{\partial x^j}
      \biggl(\sqrt{\lambda^{2n}G} \frac{1}{\lambda^2} g^{ij} \frac{\partial f}{\partial x^i}\biggr) \\
    & = \frac{1}{\lambda^n}\frac{1}{\sqrt{G}}
      \biggl(\frac{\partial\lambda^{n-2}}{\partial x^j} \sqrt{G} g^{ij}
      \frac{\partial f}{\partial x^i}
      + \lambda^{n-2}\frac{\partial}{\partial x^j}
        \biggl(\sqrt{G} g^{ij} \frac{\partial f}{\partial x^i}\biggr)
      \biggr) \\
    & = \frac{1}{\lambda^2}\Delta_g f + (n-2) \frac{1}{\lambda^3}\frac{\partial\lambda}{\partial x^j}
      g^{ij}\frac{\partial f}{\partial x^i} \\
    & = \lambda^2\biggl(\Delta_g f+(n-2)g^{ij}\frac{\partial\ln\lambda}{\partial x^j}
      \frac{\partial f}{\partial x^i}\biggr).\qedhere
  \end{align*}
\end{proof}



\begin{exercise}[6]
  设 $(M,g)$ 为黎曼流形, 若对任意 $x,y\in M$, $M$ 中从 $x$ 到 $y$ 的平行移动
  与连接 $x$ 至 $y$ 的曲线段无关, 则 $M$ 的曲率张量恒为零.
\end{exercise}

\begin{proof}
  $\forall X,Y\in\mathscr{X}(M)$, 令 $\gamma(t)$ 与 $\xi(s)$ 是连接 $x$ 与 $y$ 的曲线,
  且 $X|_{\gamma}=\gamma'(t)$, $Y|_{\xi}=\xi'(s)$. 现设 $Z\in\mathscr{X}(M)$
  即是沿 $\gamma(t)$ 的平行移动, 又是沿 $\xi(s)$ 的平行移动得到的, 即
  \[Z(b) = P_a^b(Z(a)) = \overline{P}_a^b(Z(a)),\]
  其中 $P_a^b$ 与 $\overline{P}_a^b$ 分别为沿 $\gamma(t)$ 和 $\xi(s)$
  的平行同构, 这里 $\Gamma(a)=\xi(a)=x$, 则
  \[\D_{\gamma'(a)}Z = \D_X Z(x)=0,\]
  \[\D_{\xi'(a)}Z = \D_Y Z(x)=0.\]
  $M$ 的曲率张量 $R$ 满足
  \begin{align*}
    R(X,Y)Z|_x
    & = \Bigl(\D_X\D_Y Z
      -\D_Y\D_X Z-\D_{[X,Y]}Z\Bigr)(x) \\
    & = \D_{\gamma'(a)}\D_{\xi'(s)}Z
      - \D_{\xi'(a)}\D_{\gamma'(t)}Z.
  \end{align*}
  由于 $x$ 是任意的, 从而曲率张量 $R\equiv 0$.
\end{proof}



\begin{exercise}
  设 $(M,g)$ 是 $n$ $(\geq 3)$ 维连通黎曼流形, 若对任意 $X,Y,Z,W\in\mathscr{X}(M)$ 满足恒等式
  \[R(X,Y,Z,W) = \frac{1}{n-1}\bigl(\Ric(X,W) g(Y,Z) - \Ric(X,Z) g(Y,W)\bigr).\]
  证明 $(M,g)$ 是常曲率空间.
\end{exercise}

\begin{proof}
  $\forall x\in M$, 取 $M_x$ 的幺正基 $\{e_i\}$ 使 $M$ 的 Ricci 张量对角化,
  即 $R_{ij} = \Ric(e_i,e_j) = \lambda_i\delta_{ij}$. 于是
  \begin{align*}
    R(e_i,e_j,e_k,e_l)
    & = \frac{1}{n-1}\bigl(\Ric(e_i,e_l)g_{jk} - \Ric(e_i,e_k)g_{jl}\bigr) \\
    & = \frac{\lambda_i}{n-1}\bigl(\delta_{il}\delta_{jk} - \delta_{ik}\delta_{jl}\bigr).
  \end{align*}
  由 $e_i,e_j$ 张成的截面的截面曲率为
  \begin{align*}
    K_x(e_i,e_j)
    & = -\frac{R(e_i,e_j,e_i,e_j)}{|e_i\wedge e_j|^2}
      = -\frac{\lambda_i}{n-1}\bigl(\delta_{ij}^2 - \delta_{ii}\delta_{jj}\bigr) \\
    & = \frac{\lambda_i}{n-1}.\quad(i\neq j) \\
    K_x(e_j,e_i)
    & = \frac{\lambda_j}{n-1}.
  \end{align*}
  记 $\Pi = \Span\{e_i,e_j\}$, 则
  \[K_x(\Pi) = \frac{\lambda_i}{n-1}=\frac{\lambda_j}{n-1}\Rightarrow \lambda_i=\lambda_j=\lambda(x),\]
  即 $K_x(\Pi) = \frac{\lambda(x)}{n-1}$ 与截面 $\Pi$ 无关.
  又因为 $n\geq 3$, $M$ 连通, 由 Schur 引理知, $\lambda(x)$ 是 $M$ 上的常函数,
  即 $(M,g)$ 是常曲率空间.
\end{proof}


\begin{exercise}
  设 $(M,g)$ 是 $n(\geq 3)$ 维连通黎曼流形, 且有 $\lambda\in C^{\infty}(M)$, 使得
  $\Ric = \lambda g$, 证明
  \begin{enumerate}[(1)]
    \item $M$ 是 Einstein 流形, 即 $M$ 的数量曲率为常数;
    \item 当 $n=3$ 时, $M$ 是常曲率空间;
    \item 若 $M$ 的数量曲率 $S\neq 0$, 则 $M$ 上不存在非零平行向量场.
  \end{enumerate}
\end{exercise}

\begin{proof}
  (1) 设 $\{e_i\}$ 是 $(M,g)$ 的局部标架场, $\{\omega^i\}$
  为对偶标架场, 则\footnote{下式第一个等号就是 Ricci 曲率张量的定义,
  见黎曼几何引论 Page 247--248. 讲义上给的 Ricci 曲率张量的定义是在幺正基下的特殊形式.}
  \[R_{ij} = g^{kl}R_{iklj} = \lambda g_{ij},\]
  取 trace 得 $M$ 的数量曲率
  \[S = g^{ij}R_{ij} = \lambda g^{ij}g_{ij} = n\lambda.\]

  下证 $\lambda$ 为常值函数, 当取 $\{e_i\}$ 是幺正标架时, $g_{ij}=g^{ij}=\delta_{ij}$,
  故 $R_{ij}=R_{ikkj}=\lambda\delta_{ij}$ (这里要关于指标 $k$ 求和),
  于是 $R_{ikki}=n\lambda$ (这里要关于指标 $k,l$ 求和),
  \[n e_h(\lambda) = \D_{e_h} R_{ikki} = R_{ikki,h}.\]
  由 $M$ 的第二 Bianchi 恒等式
  \begin{align*}
    0
    & = R_{ikki,h} + R_{ikih,k} + R_{ikhk,i} \\
    & = R_{ikki,h} + R_{ikih,k} + R_{kihi,k} \\
    & = R_{ikki,h} + 2R_{ikih,k} \\
    & = R_{ikki,h} - 2R_{kiih,k} \\
    & = n e_h(\lambda) - 2\D_{e_k}(\lambda\delta_{kh}) \\
    & = (n-2)e_h(\lambda).
  \end{align*}
  因为 $n\geq 3$, 故对任意 $h=1,2,\cdots,n$, 有 $e_h(\lambda)=0$,
  从而 $\diff\lambda = e_h(\lambda)\omega^h=0$,
  又 $M$ 是连通流形, 故 $\lambda$ 为常数, 从而数量曲率 $S=n\lambda$ 也为常数.

  (2) 当 $n=3$ 时, 由 $\sum_{k=1}^3 R_{ikki}=\lambda$ (固定 $i$), 得
  \[\lambda = R_{i11i}+R_{i22i}+R_{i33i} = K_{i1}+K_{i2}+K_{i3},\]
  其中 $K_{i1}$, $K_{i2}$, $K_{i3}$ 分别是由 $e_i$ 与 $e_1,e_2,e_3$
  张成的截面曲率. 于是当 $i=1$ 时有
  \[K(\Pi_{12}) + K(\Pi_{13}) = \lambda,\]
  当 $i=2$ 时,
  \[K(\Pi_{21}) + K(\Pi_{23}) = \lambda,\]
  当 $i=3$ 时,
  \[K(\Pi_{31}) + K(\Pi_{32}) = \lambda.\]
  由此得 $K(\Pi_{12}) = K(\Pi_{13}) = K(\Pi_{23})$,
  从而 $M$ 的截面曲率为常数 $\frac{\lambda}{2}$, 即 $M$ 是常曲率空间.

  (3) 假设 $M$ 上存在非零的平行向量场, 则其长度为常数, 将其单位化后仍为平行向量场, 设之为 $e$, 则
  \[\Ric(e) = \Ric(e,e) = g^{kl}R(e,e_k,e_l,e) = \lambda g(e,e) = \lambda.\]
  而
  \begin{align*}
    R(e,e_k,e_l,e)
    & = R(e_k,e,e,e_l) \\
    & = \innerp*{R(e_k,e)e}{e_l} \\
    & = \innerp*{\D_{e_k}\D_e e - \D_e\D_{e_k} e - \D_{[e_k,e]}e}{e_l} \\
    & = 0,
  \end{align*}
  故 $\lambda=0$, 这与 $S=n\lambda\neq 0$ 相矛盾,
  因此 $M$ 上不存在非零平行向量场.
\end{proof}
\section{练习题二}



\begin{exercise}[5]
  设 $\mathbb{R}_+^2=\{(x,y)\in\mathbb{R}^2 \mid y>0\}$,
  取度量 $g$ 使得 $g_{11}=g_{22}=\frac{1}{y^2}$, $g_{12}=g_{21}=0$.
  \footnote{$(\mathbb{R}^2_+,g)$ 实际上是 $H^2(-1)$ 的上半平面模型.}
  \begin{enumerate}[(1)]
    \item 求 $\mathbb{R}_+^2$ 的度量 $g$ 在坐标系 $(x,y)$ 下的 Christoffel 记号;
    \item 求 $\mathbb{R}_+^2$ 的测地线;
    \item 证明 $(\mathbb{R}^2_+,g)$ 是完备的;
    \item 求 $(\mathbb{R}^2_+,g)$ 的截面曲率;
    \item 证明 $\mathbb{R}^2_+$ 在任意一点的指数映射是微分同胚.
  \end{enumerate}
\end{exercise}

\begin{proof}
  (1) 由 Christoffel 记号的表达式
  \[\Gamma_{ij}^k = \frac{1}{2}g^{kl}\biggl(\pdv{g_{il}}{x^j}
    + \pdv{g_{jl}}{x^i} - \pdv{g_{ij}}{x^l}\biggr)\]
  以及
  \[(g_{ij})=\begin{pmatrix}
    1/y^2 & 0 \\ 0 & 1/y^2
  \end{pmatrix},\quad
  (g^{ij})=\begin{pmatrix}
    y^2 & 0 \\ 0 & y^2
  \end{pmatrix}\]
  直接算得
  \[\Gamma_{11}^1 = \Gamma_{12}^2 = \Gamma_{21}^2 = \Gamma_{22}^1 = 0,
    \quad \Gamma_{11}^2 = \frac{1}{y},\]
  \[\Gamma_{12}^1 = \Gamma_{21}^1 = \Gamma_{22}^2 = -\frac{1}{y}.\]

  (2) $(x,y)$ 已经是 $\mathbb{R}^2_+$ 的正交参数系, 即 $I=E\diff x^2+G\diff y^2$,
  其中 $E=G=\frac{1}{y^2}$, $F=0$. 设测地线 $\gamma(s)$ 与 $x$ 轴夹角为 $\theta(s)$,
  由 Liouville 公式,
  \[\begin{cases}
    \frac{\diff u}{\diff s} = \frac{1}{\sqrt{E}}\cos\theta, \\ 
    \frac{\diff v}{\diff s} = \frac{1}{\sqrt{G}}\sin\theta, \\
    \frac{\diff\theta}{\diff s} = \frac{1}{2\sqrt{G}}\frac{\partial\log E}{\partial v}\cos\theta
      - \frac{1}{2\sqrt{E}}\frac{\partial\log G}{\partial u}\sin\theta.
  \end{cases}\]
  得
  \[\begin{cases}
    \frac{\diff x}{\diff s} = y\cos\theta, \\
    \frac{\diff y}{\diff s} = y\sin\theta, \\
    \frac{\diff\theta}{\diff s} = -\cos\theta
  \end{cases}\]
  若 $\cos\theta=0$, 则 $x=C$, 此时测地线为垂直于 $x$ 轴的射线;
  若 $\cos\theta\neq 0$, 则分别将上述式子中的第一二式除以第三式得
  $\diff x=-y\diff\theta$, $\diff y=-y\tan\theta\diff\theta$, 由此解得
  \[y = c\cos\theta,\quad x = -c\sin\theta+x_0,\quad\biggl(c>0,-\frac{\pi}{2}<\theta<\frac{\pi}{2}\biggr),\]
  也即
  \[(x-x_0)^2+y^2 = c^2,\quad y>0,\]
  此时测地线为以 $(x_0,0)$ 为圆心, 以 $c$ 为半径的上半圆周.

  (3) 由 (2) 中测地线的形式知, 这些测地线可以无限延伸, 因此 $(\mathbb{R}^2_+,g)$
  是完备的黎曼流形.

  (4) 如果令 $e_1=\frac{\partial}{\partial x}$, $e_2=\frac{\partial}{\partial y}$,
  则 Gauss 曲率
  \[K = -\frac{R(e_1,e_2,e_1,e_2)}{|e_1\wedge e_2|^2}.\]
  将 (1) 中的 $\Gamma_{ij}^k$ 直接代入曲率张量表达式中得
  \begin{align*}
    R(e_1,e_2,e_1,e_2)
    & = \innerp*{R(e_1,e_2)e_1}{e_2} \\
    & = \innerp*{\D_{e_1}\D_{e_2}e_1-\D_{e_2}\D_{e_1}e_1-\D_{[e_1,e_2]}e_1}{e_2} \\
    & = \innerp*{\D_{e_1}(\Gamma_{12}^ie_i)-\D_{e_2}(\Gamma_{11}^ie_i)}{e_2} \\
    & = \frac{1}{y^4},
  \end{align*}
  且
  \[|e_1\wedge e_2|^2 = |e_1|^2|e_2|^2-\innerp*{e_1}{e_2}^2 = \frac{1}{y^4}.\]
  故 $K=-1$. 也可以代入
  \[K = -\frac{1}{\sqrt{EG}}\Biggl(\biggl(\frac{(\sqrt{E})_y}{\sqrt{G}}\biggr)_y
     + \biggl(\frac{(\sqrt{G})_x}{\sqrt{E}}\biggr)_x\Biggr) = -1.\]

  (5) 由于 $\mathbb{R}^2_+$ 为具有非正截面曲率的完备单连通黎曼流形,
  故由 Cartan-Hadamard 定理知 $\mathbb{R}^2_+$ 在任意一点的指数映射是微分同胚.
\end{proof}



\begin{exercise}
  设 $(M,\D)$ 为 $n$ 维无挠仿射联络空间, $\{e_i\}$ 为 $M$ 上的局部标架场, $\{\omega^i\}$
  为对偶标架场, 证明对任意 $\theta\in A^r(M)$, 有
  \begin{enumerate}[(1)]
    \item $\diff\theta = \sum_{i=1}^n \omega^i\wedge\D_{e_i}\theta$;
    \item $\diff\theta(X_1,\cdots,X_{r+1}) = \sum_{r=1}^{n+1}(-1)^{r+1} \bigl(\D_{X_i}\theta\bigr)
      \bigl(X_1,\cdots,\widehat{X_i},\cdots,X_{r+1}\bigr)$, 其中 $X_1\cdots,X_{r+1}\in\mathscr{X}(M)$.
  \end{enumerate}
\end{exercise}

\begin{proof}
  (1) 取法标架场 $\{e_i\}$, 使得 $\Gamma_{ij}^k(p)=0$, $\{\omega^i\}$ 为余标架场.
  令 $\theta=\theta_{i_1\cdots i_r}\omega^{i_1}\wedge\cdots\omega^{i_r}$, 
  注意到此时 $\omega_i^j=\Gamma_{ik}^j\omega^j=0$ ($\forall i,j$), 故 $\diff\omega^i=\omega^j\wedge\omega_j^i=0$, 于是
  \[\diff\theta = e_i(\theta_{i_1\cdots i_r})\omega^{i}\wedge\omega^{i_1}\wedge\cdots
    \wedge\omega^{i_r},\]
  而
  \begin{align*}
    \sum_{i=1}^n \omega^i\wedge\D_{e_i}\theta
    & = \sum_{i=1}^n \omega^i\wedge\theta_{i_1\cdots i_r,i}\omega^{i_1}\wedge\cdots\wedge\omega^{i_r} \\
    & = \sum_{i=1}^n \theta_{i_1\cdots i_r,i} \omega^i\wedge\omega^{i_1}\wedge\cdots\wedge\omega^{i_r},
  \end{align*}
  其中
  \begin{align*}
    \theta_{i_1\cdots i_r,i}
    & = e_i(\theta_{i_1\cdots i_r}) 
      - \sum_{t=1}^r \Gamma_{i_t i}^k \theta_{i_1\cdots i_{t-1}ki_{t+1}\cdots i_r} \\
    & = e_i(\theta_{i_1\cdots i_r}).
  \end{align*}
  故
  \[\diff\theta = \sum_{i=1}^n \omega^i\wedge\D_{e_i}\theta.\]

  (2) 上式两边作用在 $(X_1,\cdots,X_{r+1})$ 上即得.
\end{proof}
\section{练习题三}



\begin{exercise}[2]
  设 $x\in M$, 假设从 $x$ 出发的测地线 $\gamma$ 上所有正常 Jacobi 场都是几乎平行的,
  即存在沿 $\gamma$ 的平行单位向量场 $W(t)$, 使得满足 $U(0)=0$ 的 Jacobi 场
  $U(t)$ 可以表示为 $U(t)=f(t)W(t)$,
  其中 $f(t)$ 是定义在 $[0,b]$ 上的光滑函数, $\gamma\colon [0,b]\to M$ 是测地线.
  \begin{enumerate}[(1)]
    \item 令 $\widetilde{S}$ 是 $M_x$ 的子空间, 在 $x$ 的一个邻域内令 $S=\exp_x\widetilde{S}$.
      证明: 如果 $\gamma'(0)\in\widetilde{s}$, $\gamma([0,b])\subset S$, 则
      $P^{\gamma}$ 将 $\widetilde{S}$ 平行移动到 $\gamma(b)$ 的切空间 $S_{\gamma(b)}$.
    \item $M_x$ 中所有截面曲率是常数.
    \item 常曲率空间的 Jacobi 场都是几乎平行的.
  \end{enumerate}
\end{exercise}

\begin{proof}
  (1) 设 $\gamma(0)=x$, $\{e_i\mid i=1,\cdots,n\}$ 是 $M_x$ 的幺正基, $e_1=\gamma'(0)$.
  将 $e_i$ 沿 $\gamma(t)$ 平移得 $e_i(t)$, 其中 $e_1(t)=\gamma'(t)$.
  $\{e_i(t)\mid i=1,\cdots,n\}$ 是 $M_{\gamma(t)}$ 的幺正基. 设 $U_i(t)$ ($i\geq 2$)
  是满足 $U_i(0)=0$, $U_i'(0)=e_i$ 的沿 $\gamma$ 的正常 Jacobi 场, 由于 $M$ 的 Jacobi 场是
  几乎平行的, 故必有 $U_i(t)=f_i(t)e_i(t)$, 同时 $U_i(t)$ 还是 $\gamma(t)=\exp_x (t\gamma'(0))$
  的测地变分 $\gamma_u(t)=\exp_x t(\gamma'(0)+ue_i)$ 的变分场, 故
  \[U_i(t) = t\diff\exp_x (te_i) = f_i(t)e_i(t).\]
  这说明 $\diff\exp_x e_i\parallel e_i(t)$. 现设 $\widetilde{S}=\Span\{e_j\}$
  是 $M_x$ 的子空间, 则 $\diff\exp_x\widetilde{S}=\Span\{e_j(t)\}$,
  由定义它必是 $S=\exp_x\widetilde{S}$ 的切空间, 故 $P^{\gamma}=S_{\gamma(t)}$.

  (2) 由于 $U_i(t) = f_i(t)e_i(t)$ 是 Jacobi 场, 故
  \[f_i''(t)e_i(t) = f_i(t)R(\gamma'(t),e_i(t))\gamma'(t),\quad i\geq 2.\]
  取 $j\neq i$, 且当 $t>0$ 时, $f_i(t)\neq 0$, 则
  \[0 = f_i''(t)\innerp{e_i(t)}{e_j(t)} =
    f_i(t)\innerp*{R(\gamma'(t),e_i(t))\gamma'(t)}{e_j(t)}.\]
  故对任意 $t>0$, 有
  \[\innerp*{R(\gamma'(t),e_i(t))\gamma'(t)}{e_j(t)} = 0.\]
  令 $t\to 0$, 得 $\innerp{R(e_1,e_i)e_1}{e_j}=0$ (这里都是固定的 $i,j$).
  由 $\gamma$ 的任意性知对任意正交的向量 $X,Y,Z\in M_x$ 有
  \[R(X,Y,X,Z)=0.\]
  令 $e=\sum_{i\geq 2} \lambda_i e_i$ 是 $M_x$ 中单位向量, 则易知
  \[U(t) = \sum_{i=2}^n \lambda_i U_i(t) = \sum_{i=2}^n \lambda_i f_i(t)e_i(t)\]
  是满足 $U(0)=0$, $U'(0)=e$ 的 Jacobi 场, 由几乎平行性, 必有 $U(t)=f(t)e(t)$,
  其中 $e(t)=\sum_{i\geq 2} \lambda_ie_i(t)$, 于是
  \[f(t)e(t) = \sum_{i=2}^n \lambda_i f_i(t)e_i(t),\]
  故
  \[e(t) = \sum_{i=2}^n \lambda_i\biggl(\frac{f_i(t)}{f(t)}\biggr)e_i(t),\]
  所以
  \[0 = \D_{\gamma'}e(t) = \sum_{i=2}^n \lambda_i
    \biggl(\frac{f_i(t)}{f(t)}\biggr)'e_i(t).\]
  只要某个 $\lambda_i\neq 0$, 便有 $f_i(t)=c_if(t)$, 代入 $U_i(t)$ 的 Jacobi 方程
  得, 对该固定的 $i\neq 1$, $c_i\neq 0$, 有
  \[U_i''(t) = f_i''(t)e_i(t) = f_i(t)R(\gamma'(t),e_i(t))\gamma'(t),\]
  故
  \[c_i f''(t)e_i(t) = c_i f(t)R(\gamma'(t),e_i(t))\gamma'(t),\]
  故
  \[c_i f''(t) = -c_i f(t)K(\gamma'(t),e_i(t)),\]
  故
  \[K(\gamma'(t),e_i(t)) = -\frac{f''(t)}{f(t)},\]
  所以
  \[K(e_1,e_i) = -\lim_{t\to 0}\frac{f''(t)}{f(t)}.\]
  由 $\gamma(t)$ 的任意性, 对其他指标进行同样处理, 便有
  $\forall X,Y\in M_x$, $K(X,Y)$ 都一样, 即 $M_x$ 的所有截面曲率相同.

  (3) 设 $U(t)$ 是满足 $U(0)=0$ 的正常 Jacobi 场, 则 $U(t)$ 必能表示成
  $\{e_i(t)\mid i=2,\cdots,n\}$ 的线性组合, 设
  \[U(t) = \sum_{i=2}^n U^i(t)e_i(t).\]
  由于 $U(t)$ 满足 Jacobi 方程, 即
  \[U''(t) = \sum_{i=2}^n {U^i}''(t)e_i(t) = R(\gamma'(t),U(t))\gamma'(t).\]
  也即
  \[{U^i}''(t)e_i(t) = U^j(t)R(\gamma'(t),e_j(t))\gamma'(t),\]
  等式两边与 $e_i(t)$ 作内积得,
  \begin{align*}
    {U^i}''(t)
    & = U^j(t)\innerp*{R(e_1(t),e_j(t))e_1(t)}{e_i(t)} \\
    & = - U^j(t)c(\delta_{11}\delta_{ji} - \delta_{1i}\delta_{1j}) \\
    & = - cU^i(t),
  \end{align*}
  其中 $c$ 为 $M$ 的截面曲率. 因此每个 $U^i(t)$ ($2\leq i\leq n$)
  满足带初值条件的常微分方程
  \[\begin{cases}
    f''(t)+cf(t)=0, \\
    f(0) = 0,
  \end{cases}\]
  直接解出
  \[f(t) = \begin{cases}
    A\sinh\bigl(\sqrt{-c}t\bigr), & c<0, \\
    At, & c=0, \\
    A\sin\bigl(\sqrt{c}t\bigr), & c>0.
  \end{cases}\]
  将之统一记为 $f(t) = As(t)$, 其中 $s(t)$ 为光滑函数, 则每个 $U^i$ 可表示为 $U^i(t)=a_is(t)$,
  因此
  \[U(t) = \sum_{i=2}^n a_is(t)e_i(t) = 
    s(t)\sqrt{\sum_{i=2}^n a_i^2}\cdot\frac{1}{\sqrt{\sum_{i=2}^n a_i^2}}\sum_{i=2}^n a_ie_i(t),\]
  其中 $\frac{1}{\sqrt{\sum_{i=2}^n a_i^2}}\sum_{i=2}^n a_ie_i(t)$
  是沿 $\gamma$ 的单位平行向量场, 此即常曲率空间中的向量场是几乎平行的.
\end{proof}



\begin{exercise}
  设 $X$, $Y$ 是沿 $\gamma\colon [0,b]\to M$ 的 Jacobi 场, 证明
  \[\innerp*{X}{\D_{\gamma'}Y} - \innerp*{\D_{\gamma'}X}{Y} = C.\]
\end{exercise}

\begin{proof}
  令 $F(t)=\innerp*{X}{\D_{\gamma'}Y} - \innerp*{\D_{\gamma'}X}{Y}$,
  由于 $X$, $Y$ 均为 Jacobi 场, 它们都满足 Jacobi 方程, 即
  \[X'' = R(\gamma', X)\gamma',\quad Y'' = R(\gamma',Y)\gamma'.\]
  故
  \begin{align*}
    \frac{\diff}{\diff t}F(t)
    & = \D_{\gamma'}\Bigl(\innerp*{X}{\D_{\gamma'}Y} - \innerp*{\D_{\gamma'}X}{Y}\Bigr) \\
    & = \innerp*{\D_{\gamma'}X}{\D_{\gamma'}Y} + \innerp{X}{Y''}
      - \innerp{X''}{Y} - \innerp*{\D_{\gamma'}X}{\D_{\gamma'}Y} \\
    & = \innerp*{X}{R(\gamma',Y)\gamma'} - \innerp*{R(\gamma',X)\gamma'}{Y} \\
    & = 0.
  \end{align*}
  所以 $F(t)=C$.
\end{proof}



\begin{exercise}[5]
  设 $(M,g)$ 是黎曼流形, 给定 $O\in M$, $\rho$ 是相对于 $O$ 的距离函数. 证明
  (1) $\rho(x)$ 在 $O$ 附近不是 $C^1$ 的;
  (2) 若 $M$ 为紧流形, 则 $\rho(x)$ 在 $M\setminus\{O\}$ 上也不是 $C^1$ 的.
\end{exercise}

\begin{proof}
  (1) 取点 $O$ 的一个法坐标球邻域, $\{x^i\}$ 为法坐标系, 则
  $\rho(x)=\sqrt{\sum_{i=1}^n (x^i)^2}$, 故 $\frac{\partial\rho}{\partial x^i}=\frac{x^i}{\rho}$,
  显然极限 $\lim_{\rho\to 0}\frac{\partial\rho}{\partial x^i}$ 不存在, 故 $\rho(x)$
  在 $O$ 点附近不是 $C^1$ 的.

  (2) 如果流形是球面 $\mathbb{S}^n$, $\gamma(t)$ 是从 $O$ 出发的测地线,
  即 $\gamma(t)=\exp_O (tv)$, $v$ 是 $M_O$ 中单位向量, 由于当 $t\leq \pi$ 时,
  \[\rho(\gamma(t)) = d(O,\gamma(t)) = t,\]
  当 $t\geq\pi$ 时,
  \[\rho(\gamma(t)) = d(O,\gamma(t)) = 2\pi-t.\]
  从而 $\rho'(\pi-0)=1$, $\rho'(\pi+0)=-1$,
  这说明球面上距离函数 $\rho$ 在 $\mathbb{S}^n\setminus\{O\}$
  中是不可微的. 对于一般的紧流形, 由于 $O$ 点的切割迹内部 $\Sigma(O)$
  在指数映射下是微分同胚, 即 $\exp_O\colon \Sigma(O)\to \exp_O(\Sigma(O))$
  是微分同胚, 且 $\Sigma(O)$ 同胚于开单位球, $\Sigma(O)\cup C(O)$
  同胚于闭单位球, 故沿测地线 $\gamma(t)$ 到达割点时, 有类似于 $\mathbb{S}^n$
  的性质, 从而仍然有 $\rho(x)$ 在 $M\setminus\{O\}$ 上是不可微的.
\end{proof}



\begin{exercise}[6]
  设 $M$ 是具有正截面曲率的奇数维紧致黎曼流形, 证明 $M$ 是可定向流形.
\end{exercise}

\begin{proof}
  如果 $M$ 是不可定向流形, 对 $\forall x\in M$, 在 $M_x$ 中引入等价类,
  并令 
  \[\widehat{M}=\{(x,\mu)\mid x\in M, \mu\text{\ 是\ }M_x\text{\ 的一个定向}\}.\]
  于是 $\widehat{M}$ 是一个可定向流形, 且 $\pi\colon\widehat{M}\to M$
  是一个二重覆盖, 此时 $\pi\colon (\widehat{M},\pi^*g)\to (M,g)$
  是局部等距. $M$ 完备, 则 $\widehat{M}$ 也完备. 
  设 $\pi^{-1}(x)=\{\hat{x}_1,\hat{x}_2\}$, 其中 $\hat{x}_1=(x,\mu_1)$, $\hat{x}_2=(x,\mu_2)$,
  则 $\widehat{M}$ 中必有连接 $\hat{x}_1$ 与 $\hat{x}_2$ 的最短测地线 $\tilde{\gamma}$,
  且 $\gamma = \pi\circ\tilde{\gamma}$ 是 $M$ 中不同伦于零的以 $x$ 为基点的最短闭测地线.
  因为 $M$ 不可定向, 由 $\widehat{M}$ 的构造, 沿 $\gamma$ 必翻转 $M_x$
  的定向, 于是平移同构 $P^{\gamma}\colon M_x\to M_x$ 的行列式为 $-1$,
  而 $M$ 是奇数维的, 把 $P^{\gamma}$ 的特征值排列成
  \[\lambda_1, \overline{\lambda_1}, \cdots, \lambda_k, \overline{\lambda_k},
    -1, \cdots, -1, +1, \cdots, +1\]
  后, 特征值 $-1$ 的个数为奇数, 从而特征值 $+1$ 的个数必为偶数.
  如果 $\gamma\colon [0,b]\to M$, $\gamma(0)=\gamma(b)$, 由于
  \[P^{\gamma}\bigl(\gamma'(0)\bigr) = \gamma'(b)=\gamma'(0).\]
  从而存在另一个与 $\gamma'(0)$ 正交的单位向量 $e\in M_x$,
  使得 $P^{\gamma}(e)=e$, 即 $e(t)$ 是沿 $\gamma$ 的平行向量场,
  且 $e(0)=e(b)=e$.
  此时存在一个 $\gamma$ 的单参数闭测地线族 $\gamma_u(t)$,
  使其变分场 $U(t)$ 即为 Jacobi 场, $U(t)=e(t)$,
  这个变分的第二变分公式为
  \[L''(0) = \innerp*{\gamma'}{\D_UU}|_0^b
    + \int_0^b \Bigl(|{U^{\perp}}'|^2 + \innerp*{R(\gamma', U^{\perp}\gamma')}{U^{\perp}}\Bigr)
    \diff t.\]
  由于可以取 $\gamma_u(t)=\exp_{\gamma(t)}uU(t)$,
  故对每个 $t$, $\sigma_t(u) = \gamma_u(t)$ 是测地线, 从而 $\D_{U(t)}U(t)=0$,
  又因 $U^{\perp}=U$ 是平行向量场, 所以 ${U^{\perp}}'=0$, 于是
  \[L''(0) = \int_0^b \innerp*{R(\gamma', U^{\perp})\gamma'}{U^{\perp}}
    = -\int_0^b |\gamma\wedge U^{\perp}|^2 K(\gamma', U^{\perp})\diff t <0,\]
  这与 $\gamma$ 是以 $x$ 为基点的不同伦于零的最短闭曲线相矛盾,
  故假设不成立.
\end{proof}



\begin{exercise}[7]
  设 $(M,g)$ 是单连通完备黎曼流形, $\forall p\in M$, 若 $p$ 点沿所有从 $p$ 出发的径向测地线
  的第一共轭点都是同一点 $q$ 且 $q\neq p$, $d(p,q)=\pi$,
  证明: 如果 $K_M\leq 1$, 则 $M$ 与 $\mathbb{S}^n$ 等距.
\end{exercise}

\begin{proof}
  $\forall p\in M$, 取单位正交向量 $E_1,E_2\in M_p$,
  设 $\gamma\colon [0,\pi]\to M, \gamma(t)=\exp_x tE_1$ 是正规测地线, 且 $\gamma(0)=p$, $\gamma'(0)=E_1$,
  $q=\gamma(\pi)$ 是第一共轭点. 考虑 $\gamma$ 的变分 $\gamma_u(t)\colon [0,\pi]\times(-\epsilon,\epsilon)\to M$
  使得 $\gamma_u(t)=\exp_p t(E_1\cos u+E_2\sin u)$, 则其变分场 $J(t)$ 是 Jacobi 场,
  满足 $J(0)=J(\pi)=0$, $J'(0)=E_2$, 且
  \[J(t) = \left.\frac{\partial}{\partial u}\right|_{u=0}
    \gamma_u(t) = (\exp_p)_{*tE_1}(tE_2) = t(\exp_p)_{*tE_1} E_2.\]
  令 $E(t)=(\exp_p)_{*tE_1} E_2$, 则 $E(0)=E_2=J'(0)$, 取沿 $\gamma$ 平行的
  幺正标架场 $\{e_i(t)\}$ 使得 $e_1(t)=\gamma'(t)$, 并记
  \[J(t) = \sum_{i=2}^n J^i(t)e_i(t).\]
  则
  \begin{align*}
    0
    & = I_0^{\pi} (J,J) \\
    & = \int_0^{\pi} \Bigl(|J'(t)|^2 + \innerp*{R(\gamma'(t),J(t))\gamma'}{J}\Bigr)\diff t \\
    & = \int_0^{\pi} \biggl(\sum_{i=2}^n |{J^i}'(t)|^2 - K(\gamma'(t), J(t))|J(t)|^2\biggr)\diff t\\
    & = \int_0^{\pi} \biggl(\sum_{i=2}^n |{J^i}'(t)|^2 - 
      K(\gamma'(t), J(t))\sum_{i=2}^n |J^i(t)|^2\biggr)\diff t.
  \end{align*}
  而由 $J(0)=J(\pi)=0$ 知, 光滑函数 $J^i(t)$ 也满足 $J^i(0)=J^i(\pi)=0$,
  从而由 Wirtinger 不等式
  \[\int_0^{\pi} |{J^i}'(t)|^2\diff t\geq \int_0^{\pi} |J^i(t)|^2\diff t,\]
  得
  \[0\geq \int_0^{\pi} \bigl(1-K(\gamma'(t),J(t))\bigr)\sum_{i=2}^n |J^i(t)|^2\diff t.\]
  由于 $K(\gamma'(t),J(t))\leq 1$, $\sum_{i=2}^n |J^i(t)|^2\geq 0$ 且不恒为零, 故
  \[K(\gamma'(t), J(t))\equiv 1.\]
  注意到 $J(t)$ 与 $E(t)$ 共线, 故可令 $t\to 0$ 得
  \[K(\gamma'(t), J(t))=K(\gamma'(t), E(t)) = K(E_1,E_2)\equiv 1.\]
  由于 $p$ 点是任意的, $E_1$ 与 $E_2$ 也是任意的, 故 $K_M\equiv 1$,
  又 $M$ 是单连通完备黎曼流形, 从而由等距定理知 $M$ 等距于单位球面 $\mathbb{S}^n$.
\end{proof}
\section{练习题四}



\begin{exercise}[2]
  给出一个紧致黎曼流形的例子, 使得对某 $x\in M$, $M_x$
  中的切割迹和第一共轭轨迹均非空, 但它们互不相同.
\end{exercise}

\begin{proof}
  左图是两个半球面中间接一个长为 $4$ 的圆柱面, 右图为大半个球面再将地面光滑化.

  在 (1) 中, $\bar{x}$ 是 $x$ 的第一共轭点, 也是割点, 由于 $d(x,A)=4$,
  $d(x,\bar{x})=\pi$, $\tilde{\gamma}$ 超过 $A$ 后不是最短, 且有两条最短线连接
  $x$ 与 $B$, 而 $x$ 与 $A$ 之间只有一条最短线连接, 故 $\mathscr{L}(x)=\wideparen{AB\bar{x}}$.

  在 (2) 中, 红色测地线 $\tilde{\gamma}$ 的长度严格小于 $\pi$, 且
  $d(x,\bar{x})=L(\tilde{\gamma})=L(\gamma|_{\wideparen{xy}})$,
  此时沿 $\gamma$, $\bar{x}$ 是第一共轭点, 但 $y$ 是割点, 且连接 $x$ 与 $\bar{x}$
  的最短测地线只有 $\tilde{\gamma}$ 一条, 此时 $\mathscr{L}(x)=\wideparen{y\bar{x}}$.

  值得注意的是, 由于割点要么是第一共轭点, 要么至少有两条最短测地线, 因此 (1)
  中的第一共轭轨迹 $\mathscr{K}(x)=\wideparen{AB\bar{x}}$,
  而 (2) 中 $x$ 的第一共轭轨迹 $\mathscr{K}(x)=\{\bar{x}\}$.
\end{proof}

\begin{exercise}[7]
  设 $M$ 是紧致无边定向黎曼流形, $f\in C^{\infty}(M)$, 且 $f\geq 0$, $\Delta f\geq 0$,
  (称其为非负次调和函数, $\Delta$ 为 Beltrami-Laplace 算子), 则 $f$ 必为常数.
\end{exercise}

\begin{proof}
  由于 $M$ 紧致无边, $f\geq 0$, $\Delta f\geq 0$, 故
  \begin{align*}
    0
    & \leq \int_M f\Delta f\diff V_M = \int_M \div(f\nabla f)\diff V_M - \int_M \innerp*{\nabla f}{\nabla f}\diff V_M \\
    & = -\int_M |\nabla f|^2\diff V_M\leq 0,
  \end{align*}
  从而 $|\nabla f|=0$, $f$ 为常值函数.
\end{proof}



\begin{exercise}[9]
  在黎曼流形 $(M,g)$ 上, $f\in C^{\infty}(M)$, 求 $\Delta\bigl(|\nabla f|^2\bigr)$, 并证明
  \begin{enumerate}[(1)]
    \item 若 $M$ 是紧致无边的, $\Ric(M)\geq 0$, $\Delta f=C$, 则 $\nabla f$ 是平行向量场;
    \item 若 $\Ric(M)\geq 0$, $\Delta f=C$, $|\nabla f|=C$, 则 $\nabla f$ 是平行向量场.
  \end{enumerate}
\end{exercise}

\begin{proof}
  先推导 Weitzenb\"ock 公式的特殊形式, 对任意 $f\in C^{\infty}(M)$,
  \begin{align*}
    \Delta|\nabla f|^2
    & = \Delta\biggl(\sum_{i=1}^n f_i^2\biggr) = \biggl(\sum_{i=1}^n f_i^2\biggr)_{,jj} \\
    & = (2f_if_{i,j})_{,j} = 2f_{i,j}f_{i,j}+2f_if_{i,jj} \\
    & = 2|\D^2f|^2 + 2f_if_{j,ij}.
  \end{align*}
  由 Ricci 恒等式\footnote{见黎曼几何引论 Page 253.}
  \[f_{j,ij} = f_{j,ji} + f_kR_{jij}^k,\]
  代入得
  \begin{equation}
    \begin{aligned}
      \Delta|\nabla f|^2
      & = 2|\D^2f|^2 + 2f_i(\Delta f)_i + 2f_if_kR_{jkij} \\
      & = 2|\D^2f|^2 + 2\innerp*{\nabla f}{\nabla(\Delta f)} + 2\Ric(\nabla f,\nabla f).
    \end{aligned}\tag{$\star$}
  \end{equation}

  (1) 由于 $M$ 是紧致无边流形, 将 ($\star$) 式在 $M$ 上积分, 结合散度定理得
  \begin{align*}
    0
    & = \int_M \Delta|\nabla f|^2\diff V_M \\
    & = 2\int_M |\D^2f|^2\diff V_M +2\int_M \innerp*{\nabla f}{\nabla(\Delta f)}\diff V_M
      + 2\int_M \Ric(\nabla f,\nabla f)\diff V_M \\
    & \geq 2\int_M |\D^2f|^2\diff V_M\geq 0.
  \end{align*}
  故 $|\D^2f|^2=0\Rightarrow \D(\nabla f)=0$, 所以 $\nabla f$ 是平行向量场.

  (2) 把已知条件代入 ($\star$) 式, 即得
  \[0 = \Delta|\nabla f|^2 = 2|\D^2f|^2+2\Ric(\nabla f,\nabla f)\geq 2|\D^2f|^2\geq 0,\]
  于是 $\D^2f=0$, $\nabla f$ 为平行向量场.
\end{proof}



\begin{exercise}
  回答下列问题:
  \begin{enumerate}[(1)]
    \item 黎曼流形上的全凸子集与凸子集是否相同? 举例说明.
    \item 是否存在黎曼流形, 其上的第一共轭轨迹与割迹均非空, 但不相同.
    \item 什么样的黎曼流形上有不同伦于零的闭曲线, 使其在该流形上所有不同伦于零的分段光滑闭曲线中长度最短?
    \item 黎曼流形上到一定点 $O$ 的距离函数 $\rho(x)=d(x,O)$ 是否连续? 是否可微? 距离函数的平方 $\rho^2(x)$ 呢?
      求欧氏空间中 $\rho^2(x)$ 的 Hessian.
  \end{enumerate}
\end{exercise}

\begin{proof}
  \renewcommand{\proofname}{解}
  (1) 不一定相同, 例如在单位球面 $\mathbb{S}^2\subset\mathbb{R}^3$ 上,
  半径 $\delta\leq\frac{\pi}{2}$ 的开距离球 $U_{\delta}$ 是凸的,
  但不是全凸的.

  (2) 存在, 例如将单位球面 $\mathbb{S}^2$ 用平面截取出大半个球面, 再将底面边缘光滑化,
  此时第一共轭轨迹为 $\mathscr{K}(x)=\{\bar{x}\}$,
  割迹 $\mathscr{C}(x)$ 如图所示.

  (3) 紧致的非单连通黎曼流形.\footnote{讲义第七讲 Page 19.}

  (4) 距离函数 $\rho(x)$ 是连续的, 但不一定是可微的. 
  距离函数的平方 $\rho^2(x)$ 是可微的. 下面求 $\rho^2(x)$ 的 Hessian,
  对于任意 $X,Y\in\mathbb{R}^n$, 有
  \begin{align*}
    \D^2\rho^2(X,Y)
    & = \D^2\innerp*{x}{x}(X,Y) \\
    & = \D(2x^i\diff x^i)(X,Y) \\
    & = 2\diff x^i\otimes\diff x^i(X,Y) \\
    & = 2\innerp{X}{Y},
  \end{align*}
  故 $\Hess(\rho^2)=\D^2\rho^2=2g$, 其中 $g$ 为欧式空间中的标准度量.
\end{proof}


\end{document}